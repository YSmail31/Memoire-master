\part*{Introduction Generale}
La gestion des ressources d'une ville de manière automatique est
devenue le sujet du jour à cause de la croissance exponentiel des
habitants de villes dans le monde. Récemment, les chercheurs, de
différents domaines, sont intéressés à ce problème et continue de
proposer différentes manières pour aborder le sujet. De point de vue
informatique, les chercheurs ont commencé par le stockage et le
traitement de données via le cloud, la collecte des données par les
réseaux de capteurs sans fil (WSN) et Internet des objets (IoT),
$\cdots$. Aux contraires des approches proposées auparavant, les
systèmes de cyber physiques viennent comme une alternative qui essaye
de résoudre le problème dans sa globalité et propose de le gérer d'une
façon modulaire pour construire ce que l'on appelle ``La ville
intelligente''.

Les systèmes de cyber-physique sont des systèmes informatiques
(systèmes de cont\^ole) qui contrôlent des entités physiques tels que
des capteurs, des actionneurs, $\cdots$. La plupart des systèmes de
contr\^oles sont des systèmes qui interagissent aux stimuli et aux
changements dans l'environnement. Ainsi les mesures physiques, faites
via les capteurs, ont besoin d'être traitées dans un temps imparti
afin de réagir aux stimuli à travers les actionneurs. FADEC\cite{} est
un exemple type de ce genre de système. Le FADEC, acronyme anglais de
\emph{Full Authority Digital Engine Control}, est un système qui
s'interface entre le cockpit et le moteur d'aéronef. Il permet
d'assurer le fonctionnement d'une turbo-machine (turbo-réacteur,
turbo-propulseur, ou turbo-moteur), voire sur certains aéronefs. Le
système FADEC est un système de régulation numérique centré sur un
calculateur. Les capteurs principaux et les actionneurs sont pour leur
partie électrique dupliqués (pour des raisons de sécurité). Seuls les
organes hydrauliques (pompes, servovannes, générateurs de pression)
sont uniques (non redondants).  Il assure les fonctions de
(i)régulation de débit (alimentation en carburant, contrôle des
accélérations/décélérations), (ii) démarrage automatique transmission
des paramètres moteurs aux instruments du cockpit, (iii) gestion de la
poussée et protection des limites opérationnelles et finalement (vi)
gestion de la poussée inverse. Ainsi, ces systèmes sont classés dans
la catégorie des systèmes temps réel.

Parmi les fortes contraintes qui accompagnent les besoins temps-réel
de ces applications, il y a les contraintes sur la consommation
d'énergie. Les systèmes placés dans des endroits inaccessible et aussi
les systèmes mobiles (robots, smartphone, tablette, $\cdots$) sont
souvent alimentés par des batteries. Ceci oblige les concepteurs de
prendre en compte ce types de contraintes (temps réel \& énergie) lors
de la phase de conception afin d'augmenter la durée de vie de la
batterie le plus longtemps possible et respecter les contraintes
temps-réel.

Les processeurs modernes sont multicoeurs et permettent de réduire la
consommation d'énergie, \emph{statique} et \emph{dynamic},
principalement via deux techniques: La \textsc{DPM}\footnote{Dynamic
  Power Management} et la \textsc{DVFS}\footnote{Dynamic Voltage and
  Frequency Scaling}. La première, celle que nous utiliserons dans ce
travail, consiste à changer l'\emph{état} du matériel en éteignant des
parties du processeurs telles que les différentes horloges matérielles
(fournissant ainsi différent niveau \emph{d'endormissement} de
processeur). Cette technique permet de réduire la consommation
d'énergie statique. La deuxième technique, DVFS, consiste à changer la
fréquence opérationnelle des unités de calculs afin de réduire la
consommation d'énergie dynamique. Les architectes de processeurs
prédisent qu'en 2020 la consommation de l'énergie statique sera plus
importante que la consommation de l'énergie dynamique, plus
particulièrement les processeurs embarqués. La réduction de la
consommation d'énergie est au mépris des performances générales du
systèmes. Ceci peut êtres tolérable pour les applications non
critiques. Ce pendant, quand il s'agit de systèmes temps-réel, une
certaine minimum puissance de calcul est requise pour respecter les
contraintes temporelle du système.


Dans ce travail, il s'agit de trouver un compromis entre le minimum
requis de performance, pour \^etre ``s\^ure'' de point de vue
temporelle, et la consommation d'énergie. Nous nous intéressons plus
particulièrement à un domaine de recherche connue sous le nom ``Sleep
scheduling'' dans la communauté des systèmes temps réel.


\section*{Organisation du mémoire}

Écris ici comment ton mémoire est organisé. 
